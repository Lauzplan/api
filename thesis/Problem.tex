% Now  that  you  have  introduced  the  necessary  background 
% material  and  related  work,  you  can  explain  in  more  detail 
% the research problem that you want to address in this thesis, 
% why it is a relevant problem and how your solution would 
% advance the state of the art in this research area (by posi-
% tioning it in terms of the related work discussed in the pre-
% vious chapter). 

% Solution 
% In this chapter (or chapters), you describe in detail the solu-
% tion you have developed to address the problem. This chap-
% ter may be decomposed in several chapters describing dif-
% ferent  aspects  of  your  solution.  For  example,  one  chapter 
% introducing  a  new  formalism  you  developed,  another  de-
% scribing a novel algorithm you propose based on that for-
% malism,  and  finally  a  chapter  discussing  a  prototype  im-
% plementation of that algorithm. 
% What is most important to make clear in these chapters is 
% the conceptual ideas behind the solution. Try to describe the 
% essence of your solution so that any reader can understand 
% it,  even  though  you  can  add  some  more  detailed  sections 
% that  dive  into  the  technical  intricacies  of  the  solution  as 
% well.  A  reader  who  is  only  interested  in  the  big  picture 
% should be able to skip those more detailed sections and still 
% understand what your solution is about. A reader who wants 
% to understand your solution in full detail can decide to read 
% them. 
% Where  necessary,  provide  schemas  or  pictures  illustrating 
% how your solution works. A (good) picture often tells more 
% than  a  thousand  words.  Throughout  this  chapter,  illustrate 
% the different aspects of your solution on the running exam-
% ple. At the end of the chapter, don’t forget to position your 
% particular solution to the related work discussed in a previ-
% ous chapter.  