\section{Analysis}



\section{Methodology}

We choose to use a Agile methodology because the requirements were not well established at the start of the master thesis. By using this development methodology, we were able to change and add requirements every week, giving us more flexibility. In this case, we had two different clients:
\begin{itemize}
    \item \textbf{UCL}: the University has bought a farm where they want to experiment different market gardening type and gather data. 
    \item \textbf{Farmers}: We wanted the app to be useful not only for the university, but also for farmers in their daily life. This approach is complementary as data from farmers are useful for university's researchers.
\end{itemize}
We add several meetings with farmers at the very beginning to define the problem, their expectations and their needs

\todo[inline]{Agile parce que redéfinition régulière du probleme}

\section{Tools and Technology}

\begin{itemize}
    \item choix de Python et Django, justification
    \item Trello pour l'organisation agile des taches
    \item Github pour projet open source et versionning
    \item Travis pour continuous integration
    \item CodeClimate pour qualité du code et test coverage
\end{itemize}

